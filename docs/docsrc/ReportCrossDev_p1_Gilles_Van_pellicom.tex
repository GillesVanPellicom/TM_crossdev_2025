\documentclass{report}
\usepackage{amsmath, amssymb}
\usepackage{tikz}
\usetikzlibrary{arrows.meta, positioning, shapes, calc, fit}
\usepackage{geometry}
\usepackage{booktabs}
\usepackage{float}
\usepackage{caption}
\geometry{margin=1in}

\title{Lab 1: Qt6 Testcase Compile Using MXE\\\large YT0798 Cross Development, Thomas More}
\author{Gilles Van pellicom}
\date{\today}

\begin{document}
\maketitle

\tableofcontents
\clearpage

% ---------------------------
% Notes
% ---------------------------
\chapter*{Notes on the Process}
\addcontentsline{toc}{chapter}{Notes}

\section*{Reproducability}
\begin{itemize}
    \item \textbf{Cross-compilation environment:}
    \begin{itemize}
        \item \textbf{Platform:} Ubuntu 24.04.3 LTS Server
        \item \textbf{Hardware:} AMD EPYC 7502P, 128\,GB DDR4
        \item \textbf{Build system:} GNU Make 4.3
        \item \textbf{Project generator:} CMake 3.28.3
        \item \textbf{Compiler:} GCC 13.3.0
        \item \textbf{Cross-compile toolchain:} MXE v2023.10
    \end{itemize}

    \item \textbf{Development environment:}
    \begin{itemize}
        \item \textbf{Platform:} macOS Tahoe 26.1
        \item \textbf{Hardware:} M3 Pro, 36\,GB LPDDR5
        \item \textbf{Build system:} GNU Make 4.4
        \item \textbf{Project generator:} CMake 4.1.2
        \item \textbf{Compiler:} Apple Clang 18.0.0
        \item \textbf{IDE:} CLion 2025.2
    \end{itemize}

    \item \textbf{Testing environment:}
    \begin{itemize}
        \item \textbf{Platform:} Windows 11 24H2 (build 26100.6584)
        \item \textbf{Hardware:} Intel i7-6700K, 32\,GB DDR4
    \end{itemize}
\end{itemize}

% ---------------------------
% Testcase
% ---------------------------
\chapter{Testcase}

\section{Requirements and Fulfillment}

The test case chosen is a calculator application developed in Qt6, meeting the following evaluation criteria:

\begin{itemize}
  \item \textbf{Standard widgets:} Implemented using \texttt{QPushButton}, \texttt{QLineEdit}, and \texttt{QComboBox} for number input, operation selection, and mode switching.
  \item \textbf{Custom widget drawn with primitives:} \texttt{DisplayWidget} renders the calculator display with rounded rectangle background, custom typography, and right-aligned text using \texttt{QPainter} primitives.
  \item \textbf{Dialog for selecting a file or color:} Settings dialog includes \texttt{QColorDialog} for accent color selection.
  \item \textbf{File I/O or settings storage:} Selected accent color stored persistently using nlohmann JSON library with file I/O.
  \item \textbf{Variable content using \texttt{QStackedWidget}:} Switches between normal and scientific calculator modes, similar to Android's fragment-based navigation.
\end{itemize}

\section{Portability Considerations}

Since I usually work with portability in mind, the application required no modifications for cross-compilation with MXE. Portability practices used include but are not limited to:

\begin{itemize}
  \item \textbf{Cross-compatible JSON library:} Utilized the nlohmann JSON library, which is platform-agnostic.
  \item \textbf{Dynamic file paths:} Used \texttt{QFile f(settingsFilePath())} instead of static paths to handle file I/O portably.
  \item \textbf{OS-agnostic color picker:} Leveraged Qt's \texttt{QColorDialog} to handle color selection, allowing Qt to manage platform-specific details seamlessly.
\end{itemize}

% ---------------------------
% Chapter 1
% ---------------------------
\chapter{Setup and Dependency Installation}

\section{System Update and Package Installation}

Install all required tools for compilation of MXE and Qt6:

\begin{verbatim}
sudo apt update
sudo apt install -y autoconf automake autopoint bash bison bzip2 flex g++ \
g++-multilib gettext git gperf intltool libffi-dev libgdk-pixbuf2.0-dev \
libtool libltdl-dev libssl-dev libxml-parser-perl make python3 python3-pip
\end{verbatim}
MXE requires \texttt{mako} for \texttt{mako-render} during the Qt6 build. Install it using \texttt{pip3}:

\begin{verbatim}
pip3 install mako
\end{verbatim}
MXE assumes \texttt{python} to be the command to trigger python. Most systems have moved on to \texttt{python3}. A simple fix for this is:

\begin{verbatim}
sudo apt install python-is-python3
\end{verbatim}
Now MXE will correctly be able to exectute python scripts.

\section{Cloning and Building MXE}

Clone the MXE repository and build Qt6 for the static Windows 64-bit target:

\begin{verbatim}
git clone https://github.com/mxe/mxe.git
cd ~/mxe
make qt6 -j$(nproc)
\end{verbatim}

\section{Environment Configuration}

Add MXE tools to the PATH for access to cross-compilation utilities inside .bashrc, for persistence:

\begin{verbatim}
export PATH=~/mxe/usr/bin:$PATH
\end{verbatim}

% ---------------------------
% Chapter 2
% ---------------------------
\chapter{Project Compilation}

\section{Preamble}

Pull the project from git since the compiler machine is not my workspace. Navigate into the pulled project:

\begin{verbatim}
git clone https://github.com/GillesVanPellicom/TM_crossdev_2025/
cd ~/TM_crossdev_2025
\end{verbatim}
Create appropriate build directory:
\begin{verbatim}
mkdir build-win64-static
cd build-win64-static
\end{verbatim}

\section{CMake Configuration}

Configure the project using MXE's CMake wrapper for the \texttt{x86\_64-w64-mingw32.static} target, which sets the toolchain and locates Qt6 automatically:

\begin{verbatim}
x86_64-w64-mingw32.static-cmake ..
\end{verbatim}

\section{Building the Executable}

Compile the project to generate \texttt{crossdev.exe}:

\begin{verbatim}
make -j$(nproc)
\end{verbatim}

\section{Outcome}

The build produces \texttt{crossdev.exe}, a standalone Windows 64-bit executable statically linked with Qt6. Manual testing revealed a fully working windows executable.


A screenshot of the successful compilation is provided in the \texttt{screenshots} folder


\end{document}